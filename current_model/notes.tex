\documentclass{article}
\usepackage{amssymb}
\usepackage{amsmath}
\usepackage[ampersand]{easylist}
\usepackage{multirow} 

\begin{document}

\begin{align*}
    P(y \in C \mid x_1 \in C, x_2 \in C, \hdots, x_n \in C)
    &= \frac{P(y \in C , x_1 \in C, x_2 \in C, \hdots, x_n \in C)}{
    P(x_1 \in C, x_2 \in C, \hdots, x_n \in C)}\\
\end{align*}

But $P(x_1 \in C, x_2 \in C, \hdots, x_n \in C)=1$ since the learner is not uncertain that $x_1, \hdots, x_n$ belong to the class $C$. Then

\begin{align*}
    P(y \in C \mid x_1 \in C, x_2 \in C, \hdots, x_n \in C)
    &= P(y \in C , x_1 \in C, x_2 \in C, \hdots, x_n \in C).
\end{align*}

Since the learner is uncertain about the meaning of the class $C$, we incoropoate the eamning of a hyposthesis.

\begin{align*}
    &P(y \in C , x_1 \in C, x_2 \in C, \hdots, x_n \in C)\\
    &=\sum_h P(y \in h , x_1 \in h, x_2 \in C, \hdots, x_n \in h \mid C=h) P(C=h)\\
\end{align*}

If we consider $C$ to be a set of unweighted features, then the set of hypotheses $h$ are all possible subsets of features.
This is an intractable computation, but can be simplified if we restrict the set of hypotheses as follows:

\begin{easylist}[itemize]
    & $P(y \in h , x_1 \in h, x_2 \in C, \hdots, x_n \in h \mid C=h) =0$ if $h$ has no features in common with $x_1, \hdots, x_n$
    & $P(C=h)=0$ if 
\end{easylist}

Consider a toy example: define the following hypotheses:

\begin{align*}
    h_{sub} &= \{animal, dog, poodle\}
    \\
    h_{basic} &= \{animal, dog\}
    \\
    h_{sup} &= \{animal\}
    \\
\end{align*}

\begin{tabular}[!h]{lllll}
    \hline
    observed data & concept & hypotheses consitent & $P(y \in h , x_i \in h \mid C=h)$ & sum
    \\\hline
    1 subordinate instance &
    subordinate & 
    $h_{sub}$,
    $h_{basic}$,
    $h_{sup}$ &
    \\
    \multirow{2}{*}{$f = \{animal, dog, poodle\}$}
    & basic &
    $h_{basic}$,
    $h_{sup}$ &
    \\
    & superordinate &
    $h_{sup}$ &


\end{tabular}



\end{document}

